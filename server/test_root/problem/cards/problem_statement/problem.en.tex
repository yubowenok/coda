\problemname{Red, Black Cards}
%\illustration{.5}{filename}{Image by \href{url}{Author}}

John is playing a fancy card game with Bowen.
$N$ cards are laid on the table, facing down.
The cards are numbered from $1$ to $N$.
Each card has a color, and is either red or black.
The colors of the cards are unknown to John.
John's goal is to correctly identify the colors of all the cards without looking at any card.
For that, he asks Bowen $Q$ questions.
John's questions are numbered from $1$ to $Q$.
Each question may ask about whether a pair of cards have the same color, or what color a particular card has.
Bowen checks the colors of cards for John according to John's questions.
John thus receives a sequence of $Q$ answers, each being one of the following:

\begin{itemize}
\item {\tt d x y} : Card $x$ and card $y$ have different colors.
\item {\tt s x y} : Card $x$ and card $y$ have a same color.
\item {\tt r x} : Card $x$ has red color.
\item {\tt b x} : Card $x$ has black color.
\end{itemize}


However, since John asks so many questions, Bowen gets tired and sometimes made mistakes in answering the questions.
Fortunately, being a clever player, John can immediately catch Bowen's mistake if there is a contradiction from Bowen's answers.
When that happens, the current answer that results in a contradiction is void and the game proceeds.
The game ends immediately as soon as John gets enough information to identify the colors of all the cards.
The game also ends after $Q$ questions are all answered, even if John did not successfully identify the colors.

In this task, you will go over the questions asked by John and the answers to those questions given by Bowen.
You are to reproduce John's responses during the game.

\section*{Input}
The first line of the input has an integer $T$, the number of games played.\\
Each case has two integer $N, Q$ on the first line.\\
The next $Q$ lines describe John's questions and their answers.
Each line is in one of the four forms given above.\\
The first letter of each line is `{\tt d}', `{\tt s}', `{\tt r}' or `{\tt b}'.
If the letter is `{\tt d}' or `{\tt s}', the question asks about whether a pair of cards have the same color, and has two following integers giving a pair of card numbers.
If the letter is `{\tt r}' or `{\tt b}', the question asks about the color of a single card, and has one following integer giving the number of that card.\\
Duplicate questions may be asked, but it is possible that Bowen gave different answers to a same question (which results in contradiction).
\section*{Output}

There are two types of outputs, {\it response} and {\it result}.\\

If the answer to a question leads to a contradiction or John's success on the game, print a {\it response}:
\begin{itemize}
\item If it is a contradiction, output a single question mark ``{\tt ?}''.
Note that this answer is then ignored and not used for identifying colors.
\item If John can successfully identify the colors of all the cards, output ``{\tt I know}''.
The game then ends immediately and all the following questions are ignored.
\end{itemize}

Print the question number before each {\it response}.
Refer to the sample output for formatting details of question numbers.\\

At the end of the game, print the {\it result} of the game on a new line:
\begin{itemize}
\item If the game ends with John's success, output a string with $N$ letters.
The $i$-th letter is either `{\tt r}' or `{\tt b}' to describe the color of card $i$ being red or black.
\item If John is not able to identify the colors after all the $Q$ questions, output ``{\tt I am not sure}''.
\end{itemize}

\section*{Constraints}
\begin{itemize}
\item $1 \leq T\leq 15$
\item $1 \leq N \leq 10^5$
\item $1 \leq Q \leq 10^5$
\item $1\leq x, y \leq N, x \neq y$ for all the answers to John's questions.
\end{itemize}

\section*{Subtasks}
\begin{itemize}
\item Original constraints
\end{itemize}
